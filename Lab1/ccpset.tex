% Template for Carleton problem sets
% Author: Andrew Gainer-Dewar, 2013
% This work is licensed under the Creative Commons Attribution 4.0 International License.
% To view a copy of this license, visit http://creativecommons.org/licenses/by/4.0/ or send a letter to Creative Commons, 444 Castro Street, Suite 900, Mountain View, California, 94041, USA.
\documentclass[11pt]{article}
\usepackage{ccpset}

% The Latin Modern font is a modernized replacement for the classic
% Computer Modern. Feel free to replace this with a different font package.
\usepackage{lmodern}

\title{EE445L - Lab 01 Report}
\author{Kevin Gilbert\\ Gilberto Rodriguez}
\date{Janury 31 2014}
\prof{Professor Bard}
\course{Lab: Monday/Wednesday 5-6:15}

\begin{document}

\maketitle{}

\section*{Objective}
  • To introduce the lab equipment\\
  • To familiarize ourselves with Keil uVision4 and the ARM Cortex M processor\\
  • To develop a set of useful fixed-point output routines for use in future labs.

\section*{Lab Analysis and Discussion}

\begin{pset}
	\problem{1}
  		In what way is it good design of fixed.c that there is no arrow directly from the
        fixed.c module to the rit128x96x4.c module in the call graph for your system?\\
  
  \emph{The extra level of abstraction makes fixed.c more modular. By removing the rit128x96x4.c file directly, any errors can be limited to a particular module. It limits what any individual module is doing.}
  
	\problem{2}
    	Why is it important for the decimal point to be in the exact same physical
        position independent of the number being displayed?\\
        
     \emph{We were asked to implement a certain resolution for the numbers to be readable by humans.}
    
    \problem{3} When should you use fixed-point over floating point? When should you use
    	floating-point over fixed-point?\\
        
       \emph{ You should use fixed-point over floating point when the processor is optimized to do integer operations. You should use floating-point over fixed-point in all other cases.}

	\problem{4} When should you use binary fixed-point over decimal fixed-point? When
    	should you use decimal fixed-point over binary fixed-point? \\
        
    \emph{Binary fixed-point is supported in hardware and easier for computations. Decimal fixed-point is easier for humans to use In addition, base two uses shifts rather than multiplication/division as in base ten, making it faster to compute.}

	\problem{5} Give an example application (not mentioned in this lab assignment) for fixed-point. Describe the problem, and choose an appropriate fixed-point format. (no software implementation required).\\
    
    \emph{Say we were to develop a stopwatch application on an embedded system that did not support floating point. We can use a fixed-point data format to represent milliseconds to a human user.}
    
    \problem{6} Can we use floating point on the Arm Cortex M3? If so, what is the cost?\\
    
     \emph{There is no floating point on the Arm Cortex M3, but it can be simulated in software at the cost of multiple cycles.}
    
    \problem{Extra Credit} Is fixed-point or floating-point arithmetic faster on the
    	Pentium w/MMX?\\
        
        \emph{Fixed-point arithmetic is faster on the Pentium with the MMX because the MMX is an
        extension to the Intel Pentium chip which allows for parallel integer operation.}
\end{pset}

\section*{fixed.c}
\begin{verbatim}
/************************************************************************
  File Name: fixed.c
  Author(s): Kevin Gilbert and Gilberto Rodriguez
  Initial Creation Date: 1/22/14
  Description: A set of funtions that can take in unsigned integer, signed integer, 
	                  and binary values and output them to the LED screen on the 1968 borad
  Lab Number: 1
  TA: Mahesh Srinivasan and Zichong Li
  Date of last revision: 1/27/14
  Hardware Configuration: n/a
*************************************************************************/
#include <stdio.h>
#include <string.h>
#include <stdlib.h>
#include "fixed.h"

void Fixed_uDecOut2s(unsigned long n,  char *string){
	char holderString[10];
	if (n > 99999){
		strcpy(holderString, "***.**");
	}else{
		sprintf(holderString, "%3lu.%02lu", n / 100, n % 100);
	}
	sprintf(string, holderString);
}

void Fixed_uDecOut2(unsigned long n) {
	 char output[10];
	 Fixed_uDecOut2s(n, output);
	 printf("%s\r",output);
}

void Fixed_sDecOut3s(long n, char *string){
	char holderString[10];
	if ((n > 9999) || (n < -9999)){
		strcpy(holderString, " *.***");
	}else{
		sprintf(holderString, "%s%lu.%03lu", (n >= 0) ? (" ") : ("-"), abs(n) / 1000, abs(n) % 1000);
	}
	sprintf(string, holderString);
}

void Fixed_sDecOut3(long n){
	 char output[10];
	 Fixed_sDecOut3s(n, output);
	 printf("%s\r",output);
}

void Fixed_uBinOut8s(unsigned long n,  char *string){
	unsigned long fixedNumber = (n * 100) / 256;
  char holderString[10];
	if (n >= 256000){
	 strcpy(holderString, "***.**");}
  else{								  	
		sprintf(holderString, "%3lu.%02lu", fixedNumber / 100, fixedNumber % 100);
	}
  sprintf(string, holderString);
}

void Fixed_uBinOut8(unsigned long n){
	 char output[6];
	 Fixed_uBinOut8s(n, output);
	 printf("%s\r",output);
}
\end{verbatim}
\section*{Lab1.c}
\begin{verbatim}
#include <stdio.h>
#include <stdlib.h>
#include "fixed.h"
#include "output.h"
// const will place these structures in ROM
const struct outTestCase{       // used to test routines
  unsigned long InNumber;       // test input number
  char OutBuffer[10];           // Output String  
};
const struct s_outTestCase {
  signed long InNumber;
  char OutBUffer[10];
};

typedef const struct outTestCase outTestCaseType;
typedef const struct s_outTestCase s_outTestCaseType;
outTestCaseType outTests1[10]={ 
{     0,  "  0.00" }, //      0/100 = 0.00  
{     1,  "  0.01" }, //      1/100 = 0.01  
{    99,  "  0.99" }, //     99/100 = 0.99
{   100,  "  1.00" }, //    100/100 = 1.00
{   999,  "  9.99" }, //    999/100 = 0.99
{  1000,  " 10.00" }, //   1000/100 = 10.00
{  9999,  " 99.99" }, //   9999/100 = 99.99
{ 10000,  "100.00" }, //  10000/100 = 100.00
{ 99999,  "999.99" }, //  99999/100 = 999.99
{100000,  "***.**" }, //  error condition
};

s_outTestCaseType outTests2[10]={ 
{-100000,  " *.***"  }, //      error condition   
{ -10000,  " *.***"  }, //      error condition 
{  -9999,  "-9.999" }, //   -9999/100 = -9.999
{   -999,  "-0.999" }, //    -999/100 = -0.999
{     -1,  "-0.001" }, //      -1/100 = -0.001
{      0,  " 0.000" }, //       0/100 =  0.000
{     123, " 0.123" }, //     123/100 =  0.123
{    1234, " 1.234" }, //    1234/100 =  1.234
{    9999, " 9.999" }, //    9999/100 =  9.999
{   10000, " *.**"  }, //  error condition
};

outTestCaseType outTests3[16]={ 
{     0,  "  0.00" }, //      0/256 = 0.00  
{     4,  "  0.01" }, //      4/256 = 0.01  
{    10,  "  0.03" }, //     10/256 = 0.03
{   200,  "  0.78" }, //    200/256 = 0.78
{   254,  "  0.99" }, //    254/256 = 0.99
{   505,  "  1.97" }, //    505/256 = 1.97
{  1070,  "  4.17" }, //   1070/256 = 4.17
{  5120,  " 20.00" }, //   5120/256 = 20.00
{ 12184,  " 47.59" }, //  12184/256 = 47.59
{ 26000,  "101.56" }, //  26000/256 = 101.56
{ 32767,  "127.99" }, //  32767/256 = 127.99
{ 32768,  "128.00" }, //  32768/256 = 128
{ 34567,  "135.02" }, //  34567/256 = 135.02
{123456,  "482.25" }, // 123456/256 = 482.25
{255998,  "999.99" }, // 255998/256 = 999.99
{256000,  "***.**" }  // error
};
unsigned int Errors,AnError;
char Buffer[10];
int main(void){ // possible main program that tests your functions
  unsigned int i;
  Output_Init();
  Output_Color(15);

  Errors = 0;
  for(i=0; i<16; i++){
    printf("uBinOut8: ");
    Fixed_uBinOut8(outTests3[i].InNumber);
  }
  printf("________________\r");
  Delay(8000000);

  for(i=0; i<10; i++) {
	printf("sDecOut3: ");
	Fixed_sDecOut3(outTests2[i].InNumber);
  }
  printf("________________\r");
  Delay(8000000);
  for(i=0;i<10;i++) {
	printf("uDecOut2: ");
	Fixed_uDecOut2(outTests1[i].InNumber);
  }
  Delay(8000000);
    if(strcmp(Buffer, outTests3[i].OutBuffer)){
      Errors++;
      AnError = i;
    }	
  for(;;) {} /* wait forever */
}

\end{verbatim}
\end{document}