% Template for Carleton problem sets
% Author: Andrew Gainer-Dewar, 20131
\documentclass[twoside]{article}
\usepackage{ccpset}
\usepackage{graphicx, pdfpages}
\usepackage{fixltx2e}

% The Latin Modern font is a modernized replacement for the classic
% Computer Modern. Feel free to replace this with a different font package.
\usepackage{lmodern}

%\titleformat{\subsection}[runin]{}{}{}{}[]

\title{EE445L - Lab 08 Report}
\author{Kevin Gilbert\\ Gilberto Rodriguez}
\date{April 4, 2014}
\prof{Professor Bard}
\course{Lab: Monday/Wednesday 5-6:15}

\begin{document}
\raggedbottom
\maketitle{}

\section{Requirements Document}
\subsection{Overview}
\subsubsection{Objectives}
Our project will be centered around developing a board to form the basis of a teleoperated car. The primary goal is to develop an RC car that will communicate wirelessly with a ZigBee and use onboard sensors to allow a level of self-control.
\subsubsection{Roles and Responsibilities}
This device is aimed towards DIY and hobbyist groups, as well as high school and college level robotics teams. Gilbert and I will design the circuit schematic and software design layout as a group. PCB routing will be handled primarily by a single person, as it is difficult to share work during this process. The software realization will be written by both of us as well.
\subsubsection{Interactions with Existing Systems}
We will be using the LM3S1968 board as a controller for our device, using a ZigBee.
\subsection{Function Description}
\subsubsection{Functionality}
The system will have an on-board LM3S811 chip to collect data from the ZigBee and interface with the motors. The embedded device will also include motor controllers for actuation, and onboard sensors to allow a degree of autonomity. A power regulator will allow for battery operation.
\subsubsection{Performance}
ISR lengths through debugging instruments. Current needed to power board with and without motors running.
\subsubsection{Usability}
The LM3S1968 will be used to broadcast the wireless signal to the car. User input will be captured using button inputs, and the car's speed will displayed through the onboard OLED.
\subsection{Deliverables}
\subsubsection{Reports}
We will write a report for Labs 8.
%\subsubsection{Outcomes}
%\bf{Lab07:}
%\begin{enumerate}
%\item 1-page requirements document
%\item Hardware Design: Regular circuit diagram (SCH file) PCB layout and three printouts (top, bottom and combined)
%\item Software Design: Include the requirements document (Preparation a)
%\item Measurement Data: Give the estimated current (Procedure d). Give the estimated cost (Procedure e)
%\item Analysis and Discussion (none)
%\end{enumerate}

%\noindent \bf{Lab11:}
%\begin{enumerate}
%\item Objectives: 2-page requirements document
%\item Hardware Design :Detailed circuit diagram of the system (from Lab 7)
%\item Software Design (no software printout in the report): Briefly explain how your software works (1/2 page maximum)
%\item Measurement Data: As appropriate for your system. Explain how the data was collected.
%\end{enumerate}

\section{Hardware Design}
\subsection{Circuit Schematics}
		\subsubsection{Control Circuit}
			\includegraphics[width=\textwidth]{circuitDiagram}
		%\subsubsection{Power Circuit}
			%\includegraphics[width=\textwidth]{powerCircuit}

 
\end{document}